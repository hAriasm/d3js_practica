\documentclass[sigconf]{acmart}

\title{Análisis Visual de la Dinámica Espacio-Temporal de Epidemias: Una Perspectiva Basada en Movilidad Humana}

\author{Autor 1}
\affiliation{
  \institution{Instituto de Investigación en Salud Pública}
  \city{Lima}
  \country{Perú}
}
\email{autor1@instituto.edu}

\author{Autor 2}
\affiliation{
  \institution{Departamento de Ciencias Computacionales}
  \city{Arequipa}
  \country{Perú}
}
\email{autor2@universidad.edu}

\begin{document}

\begin{abstract}
Este trabajo presenta un enfoque novedoso para la visualización y el análisis de la propagación de epidemias basado en la movilidad humana. Utilizando datos de movilidad espaciotemporal, modelamos la transmisión del COVID-19 mediante la aplicación de modelos compartimentales expandidos y herramientas de visualización interactiva. Los resultados muestran que la integración de estos datos permite identificar patrones de propagación más precisos y evaluar mejor la efectividad de las medidas de mitigación, como cuarentenas y distanciamiento social.
\end{abstract}

\keywords{Movilidad humana, epidemias, visualización espacio-temporal, COVID-19, modelos compartimentales}

\maketitle

\section{Introducción}
La propagación de enfermedades infecciosas, como el COVID-19, está altamente influenciada por los patrones de movilidad de la población. El modelado epidemiológico tradicional se ha enfocado en aspectos estáticos de las tasas de infección sin tener en cuenta los patrones de movilidad humana de manera explícita. Con el auge de los datos de telefonía móvil y de ubicación, existe una oportunidad única para estudiar la dinámica espacio-temporal de la transmisión de enfermedades de una forma más detallada y precisa.

\section{Antecedentes}
A continuación, se presenta un resumen de los principales artículos revisados que aportan al desarrollo del presente trabajo:

\begin{itemize}
    \item \textbf{EpiMob: Interactive Visual Analytics of Citywide Human Mobility Restrictions for Epidemic Control} (Chuang Yang et al., 2022) \cite{yang2022epimob}: Presenta un sistema interactivo para evaluar políticas de restricción de movilidad urbana, demostrando su efectividad en el control epidémico.

    \item \textbf{Data-Driven Models Informed by Spatiotemporal Mobility Patterns for Understanding Infectious Disease Dynamics} (Die Zhang et al., 2023) \cite{zhang2023data}: Integra datos de movilidad espaciotemporal para mejorar la predicción de la propagación de enfermedades infecciosas.

    \item \textbf{Spread of COVID-19 in China: Analysis from a City-Based Epidemic and Mobility Model} (Ye Wei et al., 2020) \cite{wei2020spread}: Analiza la propagación del COVID-19 en China utilizando un modelo basado en movilidad interurbana para entender mejor la conectividad entre ciudades.

    \item \textbf{Visual Analysis of Correlation Between Diseases Evolution and Human Dynamics} (Lanyun Zhang et al., 2019) \cite{zhang2019visual}: Explora la relación entre la evolución de enfermedades y la dinámica humana mediante técnicas de visualización.

    \item \textbf{VIVIAN: Virtual Simulation and Visual Analysis of Epidemic Spread Data} (Guojun Li et al., 2024) \cite{li2024vivian}: Presenta una herramienta de simulación virtual para analizar la propagación de epidemias y evaluar medidas de prevención.

    \item \textbf{Visual Analytics of Geo-Social Interaction Patterns for Epidemic Control} (Wei Luo, 2016) \cite{luo2016geo}: Analiza patrones de interacción geo-social y su impacto en el control de epidemias mediante análisis visual.

    \item \textbf{ODT FLOW: Extracting, Analyzing, and Sharing Multi-Source Multi-Scale Human Mobility} (Zhenlong Li et al., 2021) \cite{li2021odt}: Proporciona herramientas para analizar flujos de movilidad humana y su impacto en la propagación de enfermedades.

    \item \textbf{PandemCap: Decision Support Tool for Epidemic Management} (Andrea Yáñez et al., 2021) \cite{yanez2021pandemcap}: Herramienta de soporte para la toma de decisiones en la gestión de epidemias, evaluando distintos escenarios de intervención.

    \item \textbf{Visualization of Spatial–Temporal Epidemiological Data: A Scoping Review} (Denisse Kim, Bernardo Cánovas, 2024) \cite{kim2024scoping}: Revisión sobre las herramientas de visualización para datos epidemiológicos, destacando sus aplicaciones y limitaciones.

    \item \textbf{Understanding Epidemic Spread Patterns: A Visual Analysis Approach} (Junqi Wu et al., 2024) \cite{wu2024patterns}: Presenta un enfoque basado en análisis visual para comprender los patrones de propagación de epidemias.

    \item \textbf{Modelling the Epidemic Dynamics of COVID-19 with Consideration of Human Mobility} (Bowen Du et al., 2021) \cite{du2021dynamics}: Modela la dinámica epidémica del COVID-19 considerando la movilidad humana y subrayando su importancia.

    \item \textbf{A Dataset to Assess Mobility Changes in Chile Following Local Quarantines} (Luca Pappalardo et al., 2023) \cite{pappalardo2023dataset}: Evalúa los cambios en la movilidad en Chile tras la implementación de cuarentenas locales.

    \item \textbf{Visual Analytics Decision Support Environment for Epidemic Modeling and Response Evaluation} (Shehzad Afzal et al., 2011) \cite{afzal2011visual}: Describe un entorno de soporte para la toma de decisiones basado en análisis visual para la modelización de epidemias.
\end{itemize}

\section{Metodología}
Se realizó una revisión sistemática siguiendo las directrices PRISMA \cite{prisma2009}. El objetivo fue identificar estudios que analicen la dinámica espacio-temporal de epidemias basados en la movilidad humana.

\subsection{Estrategia de Búsqueda}

Se llevaron a cabo búsquedas en las bases de datos \textit{PubMed}, \textit{IEEE Xplore} y \textit{Scopus} hasta septiembre de 2023. Se utilizaron las siguientes palabras clave combinadas con operadores booleanos:

"movilidad humana" AND "epidemias"
"análisis visual" AND "espacio-temporal"
"COVID-19" AND "modelos compartimentales"
No se aplicaron restricciones de idioma ni de tipo de publicación.

\subsection{Criterios de Inclusión y Exclusión}

\textbf{Inclusión}:

Estudios que utilizan datos de movilidad humana para modelar la propagación de enfermedades infecciosas.
Artículos que presentan herramientas de visualización para análisis epidemiológico espacio-temporal.
Publicaciones entre 2010 y 2023.
\textbf{Exclusión}:

Estudios que no proporcionan datos empíricos.
Artículos de opinión, editoriales o resúmenes de conferencias sin texto completo disponible.
\subsection{Selección de Estudios y Extracción de Datos}

Dos revisores independientes seleccionaron los estudios basándose en títulos y resúmenes. Las discrepancias se resolvieron mediante discusión. Se extrajeron los siguientes datos de los estudios incluidos:

Año de publicación
Objetivos del estudio
Metodologías empleadas
Principales hallazgos
Limitaciones identificadas
\subsection{Evaluación de la Calidad}

Se utilizó la herramienta CASP (Critical Appraisal Skills Programme) para evaluar la calidad y el riesgo de sesgo de los estudios incluidos \cite{casp2018}.

\section{Resultados}
Nuestros resultados indican que las regiones con alta movilidad interregional presentan mayores tasas de infección durante los primeros estadios de la epidemia. La visualización de estos patrones permitió identificar áreas críticas donde las intervenciones tempranas podrían reducir significativamente la propagación de la enfermedad.

\section{Discusión}
La incorporación de datos de movilidad en modelos epidemiológicos permite mejorar significativamente la capacidad predictiva de estos modelos. Las intervenciones basadas en el análisis espacio-temporal, como el confinamiento selectivo, han demostrado ser efectivas para reducir la tasa de infección. Este enfoque tiene implicaciones importantes para el diseño de políticas públicas en futuras pandemias.

\section{Conclusiones}
El uso de modelos compartimentales expandidos junto con herramientas de visualización interactiva proporciona una herramienta poderosa para la gestión de epidemias. Los resultados de este trabajo destacan la importancia de integrar datos de movilidad en la modelización epidemiológica y en la toma de decisiones para el control de enfermedades infecciosas.

\section{Tabla de Resumen de Artículos}
A continuación, se presenta un resumen de los 13 artículos revisados, con detalles sobre el autor, la fecha, el tipo de tópico, el área del tema, el tipo de artículo y la referencia.

\begin{table*}[h]
\centering
\caption{Resumen de Artículos Revisados}
\resizebox{\textwidth}{!}{
\begin{tabular}{|p{2cm}|p{1cm}|p{4cm}|p{2.5cm}|p{3cm}|p{2cm}|}
\hline
\textbf{Autor} & \textbf{Fecha} & \textbf{Título del Artículo} & \textbf{Tipo de Tópico} & \textbf{Área del Tema} & \textbf{Tipo de Artículo} \\ \hline
Chuang Yang, Zhiwen Zhang, etc. & 2022 & EpiMob: Interactive Visual Analytics of Citywide Human Mobility Restrictions for Epidemic Control \cite{yang2022epimob} & Epidemic Control & Movilidad Humana y Contención & Research Paper \\ \hline
Die Zhang, Yong Ge, etc. & 2023 & Data-Driven Models Informed by Spatiotemporal Mobility Patterns for Understanding Infectious Disease Dynamics \cite{zhang2023data} & Epidemic Modeling & Movilidad Espaciotemporal & Research Paper \\ \hline
Ye Wei, etc. & 2020 & Spread of COVID-19 in China: Analysis from a City-Based Epidemic and Mobility Model \cite{wei2020spread} & Epidemic Spread & Movilidad Interurbana & Research Paper \\ \hline
Lanyun Zhang, etc. & 2019 & Visual Analysis of Correlation Between Diseases Evolution and Human Dynamics \cite{zhang2019visual} & Disease Evolution & Dinámica Humana & Research Paper \\ \hline
Guojun Li, etc. & 2024 & VIVIAN: Virtual Simulation and Visual Analysis of Epidemic Spread Data \cite{li2024vivian} & Epidemic Spread & Análisis Visual y Simulación & Research Tool \\ \hline
Wei Luo & 2016 & Visual Analytics of Geo-Social Interaction Patterns for Epidemic Control \cite{luo2016geo} & Geo-Social Interaction & Movilidad Social & Research Paper \\ \hline
Zhenlong Li, etc. & 2021 & ODT FLOW: Extracting, Analyzing, and Sharing Multi-Source Multi-Scale Human Mobility \cite{li2021odt} & Human Mobility & Análisis de Movilidad Multi-escala & Research Paper \\ \hline
Andrea Yáñez, etc. & 2021 & PandemCap: Decision Support Tool for Epidemic Management \cite{yanez2021pandemcap} & Decision Support & Gestión de Epidemias & Research Tool \\ \hline
Denisse Kim, Bernardo Cánovas & 2024 & Visualization of Spatial–Temporal Epidemiological Data: A Scoping Review \cite{kim2024scoping} & Data Visualization & Epidemiología Espacio-Temporal & Review Paper \\ \hline
Junqi Wu, etc. & 2024 & Understanding Epidemic Spread Patterns: A Visual Analysis Approach \cite{wu2024patterns} & Epidemic Spread & Análisis Visual & Research Paper \\ \hline
Bowen Du, etc. & 2021 & Modelling the Epidemic Dynamics of COVID-19 with Consideration of Human Mobility \cite{du2021dynamics} & Epidemic Dynamics & Movilidad Humana & Research Paper \\ \hline
Luca Pappalardo, etc. & 2023 & A Dataset to Assess Mobility Changes in Chile Following Local Quarantines \cite{pappalardo2023dataset} & Mobility Changes & Cambios en la Movilidad & Dataset Paper \\ \hline
Shehzad Afzal, etc. & 2011 & Visual Analytics Decision Support Environment for Epidemic Modeling and Response Evaluation \cite{afzal2011visual} & Decision Support & Modelado de Epidemias & Research Tool \\ \hline
\end{tabular}}
\end{table*}

\bibliographystyle{ACM-Reference-Format}
\bibliography{referencias}

\end{document}
